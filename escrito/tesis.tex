\documentclass[12pt,a4paper]{article}
\usepackage[utf8]{inputenc}
\usepackage[spanish]{babel}
\usepackage{amsmath}
\usepackage{amsfonts}
\usepackage{amssymb}
\usepackage{graphicx}
\usepackage{float}
\usepackage{listings}
\usepackage{xcolor}
\usepackage{hyperref}
\usepackage{geometry}
\usepackage{booktabs}
\usepackage{array}
\usepackage{longtable}

% Configuración de geometría
\geometry{margin=2.5cm}

% Configuración para reducir espacio en figuras
\setlength{\textfloatsep}{10pt}
\setlength{\floatsep}{10pt}
\setlength{\intextsep}{10pt}

% Configuración de hyperref
\hypersetup{
    colorlinks=true,
    linkcolor=blue,
    filecolor=magenta,      
    urlcolor=cyan,
    citecolor=red
}

% Configuración de listings para código
\lstset{
    basicstyle=\ttfamily\small,
    breaklines=true,
    frame=single,
    numbers=left,
    numberstyle=\tiny,
    keywordstyle=\color{blue},
    commentstyle=\color{green!60!black},
    stringstyle=\color{red},
    backgroundcolor=\color{gray!10}
}

% Información del documento
\title{\textbf{Desarrollo de un Sistema Modular para el Análisis de Gotas Cargadas Eléctricamente en Tormentas}}
\author{Estudiante de Licenciatura en Ciencias de la Computación}
\date{\today}

\begin{document}

\maketitle

\begin{abstract}
    Despues lo hago al abstract
\end{abstract}

\tableofcontents
\newpage

\section{Introducción}

Las gotas de lluvia pueden transportar carga eléctrica, la cual se adquiere principalmente a través de procesos microfísicos dentro de la nube, como las colisiones entre distintos hidrometeoros (granizo, graupel, copos de nieve, cristales de hielo). El signo y la magnitud de estas cargas están estrechamente relacionados las condiciones internas de la tormenta, como la temperatura, el viento, las particular presentes, etc. Al mismo tiempo, la distribución de tamaños de las gotas brinda información sobre la microfísica de la precipitación y resulta esencial para estimar la lluvia, alimentar modelos numéricos, interpretar mediciones de radares y estudiar procesos como la erosión de suelos o crecidas.

En este sentido, medir y analizar simultáneamente el tamaño y la carga de las gotas resulta fundamental, ya que nos permite recopilar información que no solo ayuda a comprender la física de las tormentas, sino que también es útil para mejorar herramientas de pronóstico y para el estudio de fenómenos atmosféricos.

En este marco, ya existen instrumentos capaces de realizar estas mediciones, así como programas para procesarlas. Sin embargo, el código disponible actualmente presenta limitaciones tanto en diseño como en rendimiento: está poco modularizado, resulta difícil de mantener y su desempeño es insuficiente para procesar de manera eficiente grandes volúmenes de datos. En esta tesis se propone un nuevo sistema de análisis que resuelve esas limitaciones, mejorando la eficiencia, la escalabilidad y la mantenibilidad del software, y facilitando así la obtención de resultados más completos y confiables.

\subsection{Instrumento de Medición}

El dispositivo utilizado consiste en un anillo de inducción de latón de 6 cm de diámetro y 1.5 cm de longitud, posicionado a 5.7 cm por encima de una placa plana de aluminio de 20 cm de diámetro. Tanto el anillo como la placa están conectados eléctricamente a amplificadores de corriente de alta ganancia con una amplificación de 5 $\times$ 10$^8$ V/A. Para proteger contra interferencias electromagnéticas, el anillo, la placa y los amplificadores están encerrados dentro de un contenedor metálico que actúa como jaula de Faraday. La figura \ref{fig:instrumento_medicion} muestra el dispositivo.

\begin{figure}[!hb]
    \centering
    \includegraphics[width=0.7\textwidth]{figures/instrumento_de_medicion.png}
    \caption{(a) Diagrama del dispositivo de medición; (b) circuito amplificador inversor de corriente; (c) fotografía del dispositivo en el techo de la Facultad de Matemática, Astronomía, Física y Computación, Universidad Nacional de Córdoba.}
    \label{fig:instrumento_medicion}
\end{figure}

Los amplificadores están protegidos contra daños por agua al estar asegurados dentro de un recinto impermeable. La jaula de Faraday presenta una abertura ligeramente mayor que el anillo, permitiendo que las gotas de lluvia entren únicamente a través del anillo de inducción. Toda el agua que llega a la placa se drena a través de sus lados y se recolecta en un contenedor de aluminio, que también está situado dentro de la jaula de Faraday pero eléctricamente aislado de ella.

Las gotas cargadas eléctricamente inducen corrientes tanto en el anillo como en la placa. Al caer una gota, primero se acerca al anillo. Al hacerlo, se induce una corriente de la polaridad opuesta a la carga de la gota. Luego, al alejarse, esta polaridad se invierte. Mientras tanto, la gota se acerca a la placa, induciendo también una corriente de polaridad opuesta a la carga de la gota, lo cual culmina en una meseta en la corriente dado que la placa de aluminio absorbe el impacto de la gota y se le transfiere toda la carga a la placa, la cual se disipa en el tiempo. La figura \ref{fig:corriente_gotas} muestra la dinamica de este proceso con el registro de una gota.

\begin{figure}[!hb]
    \centering
    \includegraphics[width=0.7\textwidth]{figures/corriente_gotas.png}
    \caption{Corriente inducida por una gota cargada negativamente en el anillo (puntos solidos) y la placa (puntos huecos).}
    \label{fig:corriente_gotas}
\end{figure}

La adquisición de datos se realiza a una tasa de 5 kHz por canal (5000 datos por segundo). Por limitaciones de hardware, la adquisición de datos no se puede realizar al mismo tiempo que la escritura a disco, por lo que cada segundo se pierden $\sim$50ms de datos. Esto debe ser tenido en cuenta luego para analizar los datos obtenidos.

\subsection{Procesamiento de Señales}

El procesamiento transforma las señales crudas registradas por el instrumento en una lista de gotas con sus propiedades. Para ello se emplean dos programas que se ejecutan de manera secuencial: el primero realiza un preprocesamiento de las señales y el segundo extrae las gotas y calcula sus características. El flujo general consta de cinco pasos principales:

\begin{enumerate}
    \item Rellenado de huecos en los datos.
    \item Remoción del offset de las señales.
    \item Búsqueda de pulsos en las señales.
    \item Aplicación de filtros de calidad para extraer las gotas.
    \item Cálculo de propiedades de las gotas.
\end{enumerate}

En este contexto se denomina pulso a la variación característica de corriente producida por el paso de una única gota a través del anillo y su impacto en la placa.

Para el procesamiento de los datos se han desarrollado dos programas en Fortran que trabajan de manera secuencial.

El primer programa se encarga del preprocesamiento de los datos, específicamente del rellenado de huecos mediante interpolación lineal y de la remoción del offset de las señales.

El segundo programa realiza el análisis principal de las señales ya pre-procesadas. Implementa el algoritmo de detección de pulsos, aplica filtros para separar las gotas válidas y calcula sus propiedades físicas, incluyendo tamaño, carga eléctrica y velocidad de caída.

\subsection{Necesidad de Mejoras}

El código actual, aunque funcional, sufre de varios problemas tanto en su diseño como en su rendimiento.

En primer lugar, el código consta de pocos archivos con una cantidad de líneas muy grande, lo que dificulta su mantenimiento y modificación. Cualquier cambio requiere revisar y entender todo el código, ya que las responsabilidades no están bien asignadas. Además, hay muchas variables sin nombres descriptivos y constantes hard-codeadas sin referencia alguna, lo que complica aún más la comprensión.

En segundo lugar, para grandes volúmenes de datos, la ejecución es muy lenta. Por ejemplo, para procesar datos de una tormenta de 5 horas de duración (aproximadamente 100M de datos), el programa tarda alrededor de 6 horas en completarse. A esto se suma un consumo de memoria excesivamente alto, lo que obliga a procesar los datos de a partes. A su vez el algoritmo actual no tiene en cuenta las gotas que quedan en medio de estos cortes, resultando en la pérdida de algunos datos.

\subsection{Objetivos}

\subsubsection{Objetivo General}

Desarrollar un sistema modular y automatizado para el análisis de datos de gotas cargadas eléctricamente que mejore significativamente la eficiencia, mantenibilidad y escalabilidad del código existente, facilitando la obtención de nuevos resultados.

El nuevo sistema debe ser capaz de procesar grandes volúmenes de datos de manera eficiente y, como mínimo, detectar la misma cantidad de gotas que el código original, intentando incrementarla en lo posible.

\subsubsection{Objetivos Específicos}

\begin{enumerate}
    \item \textbf{Diseñar una arquitectura modular} que separe las etapas del procesamiento en al menos tres componentes independientes y reutilizables.

    \item \textbf{Implementar procesamiento paralelo} para analizar múltiples archivos de datos simultáneamente, con el objetivo de reducir el tiempo total de análisis de una tormenta de 5 horas de 6 horas a aproximadamente 15 minutos.

    \item \textbf{Desarrollar un sistema de automatización} que permita ejecutar la cadena completa de análisis con un único comando y sin intervención manual.

    \item \textbf{Optimizar el rendimiento y el uso de memoria} del algoritmo de detección de gotas para manejar aproximadamente 100 millones de muestras utilizando menos de 2 GB de memoria y asegurando al menos la misma cantidad de gotas detectadas que el programa original.

    \item \textbf{Documentar completamente} el sistema mediante un manual de usuario y comentarios en el código para facilitar su uso y mantenimiento futuro.
\end{enumerate}

\subsection{Estructura de la Tesis}

Despues la hago

\section{Procedimiento}

\subsection{Análisis del Código Existente}

Aca iria el analisis de todo el codigo existente. Señalando todas las fallas especificas del codigo existente. Tambien habria que señalar el error que descubrimos en el llenado de huecos (que se hacia constante y no se hacia la interpolacion lineal).

\subsection{Algoritmos y estructuras de datos utilizadas}

Aca explicaria min-max-queue que lo uso para detectar los pulsos en O(1).

Tambien explicaria el algoritmo de promedio movil utilizado para quitar el offset de las señales.

Y capaz que me estoy olvidando de algo mas

\subsection{Implementación en C++}

Toda la explicacion del codigo de C++

\subsection{Sistema de Automatización}

Toda la explicacion del codigo de Python

\section{Resultados}

- Comparativas de tiempo
- Comparativas de memoria
- Comparativas de cantidad de gotas detectadas

Señalar todos los objetivos cumplidos en base a los planteados

\subsection{Limitaciones y Áreas de Mejora}

Despues vemos

\section{Conclusiones}

Veremos 

\section{Referencias}

\begin{enumerate}
    \item Despues las hago
\end{enumerate}

\end{document}
